%!TEX root = ../../Master.tex
\subsubsection{Block} % (fold)
\label{subsub:block}

Block er en klasse til håndtering af en klods. Klassen har ansvar for at holde styr på klodsens orientering samt placering i form af et koordinat for klodsens centrum. Dette opnås ved hjælp af klassens to funktioner, som er beskrevet herunder. \\

\textbf{img2base\_transform} \\
Denne funktion har til formål at transformere koordinatsættet fra billedet, som angives i pixels til koordinatsættet for robotten, som angives i cm.\\

Første del af funktionen står for at oversætte pixelkoordinater til koordinator i cm. Dette er gjort ved at finde antallet af pixels på en cm på billedet og derved dele alle pixel koordinator med dette. \\

Den sidste del består i at transformere billedkoordinatsættet til robotkoordinatsættet. De to koordinatsæt er illustret i \autoref{fig:img2base}, hvor koordinatsættet øverst i venstre side er for billedet og koordinatsættet nederst i midten er for robotten. Som figuren indikerer, kræver denne transformation både translation i x-aksens og y-aksens retning samt en rotation om både x- og z-aksen. \\ 

\fixme{Ændre 'Figure' til 'Figur'}

\begin{figure}[h]
\centering
\includegraphics[scale=0.4]{images/img2base}
\caption{Billedkoordinatsæt og robotkoordinatsæt i forhold til bord.}
\label{fig:img2base}
\end{figure}

I implementeringen af dette oprettes derfor to matricer - en transformationsmatrix med både translation og rotation og en rotationsmatrix for den sidste rotation. Dette kan ses i \autoref{transformation}. Matrix A translaterer koordinaterne langs x- og y-aksen og roterer disse omkring x-aksen. Matrix B roterer koordinaterne om z-aksen. For at anvende disse matricer, skal koordinattet, som skal transformeres, være et homogent koordinat.

\begin{lstlisting}[caption=Matricer til transformation., label=transformation, language=Python]
A = np.matrix([
    [1, 0,           0,            tx],
    [0, cos(thetax), -sin(thetax), ty],
    [0, sin(thetax), cos(thetax),  tz],
    [0, 0,           0,            1 ]
    ])

B = np.matrix([
    [cos(thetaz),  sin(thetaz), 0, 0],
    [-sin(thetaz), cos(thetaz), 0, 0],
    [0,            0,           1, 0],
    [0,            0,           0, 1]
    ])
\end{lstlisting}

\textbf{find\_orientation} \\
Denne funktion har til formål at bestemme orienteringen på klodsen i forhold til robottens x-akse. Robotten har to orienteringer, da den har to ikke parallelle sider. Dette er illustreret på \autoref{fig:find_orientation}. \\

\fixme{Ændre 'Figure' til 'Figur'}

\begin{figure}[h]
\centering
\includegraphics[scale=0.4]{images/findOrientation}
\caption{Blokkens orientering i forhold til robottens x-akse.}
\label{fig:find_orientation}
\end{figure}

Fra den tidligere billedprocessering kendes klodsens fire hjørner. Ud fra disse fire hjørner fås to vektorer, som har samme orientering som klodsens to sider. Vinklen mellem disse to vektorer og x-aksen er orienteringen. Vektoren findes ved at trække de to punkter fra hinanden og viklen opnås ved at bruge tangens på $y/x$. Implementationen af dette ses i \autoref{orientation}.

\begin{lstlisting}[caption=Funktion til at finde orientering på en klods., label=orientation, language=Python]
def find_orientations(self, block_corners):
    thetas = []

    for i in range(0, 2):
        corner1 = self.img2base_transform(block_corners[i][0], block_corners[i][1])
        corner2 = self.img2base_transform(block_corners[i+1][0], block_corners[i+1][1])

        dx = corner2[0] - corner1[0]
        dy = corner2[1] - corner1[1]

        theta = atan2(dy, dx)

        thetas.append(theta)

    return thetas
\end{lstlisting}

% subsubsection block (end)